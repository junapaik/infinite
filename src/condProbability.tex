%%%%%%%%%%%%%%%%%%%%%%%%%%%%%%%%%%%%%%%%%%%%%%%%%%%%
\begin{edXchapter}{Conditional Probability}
%%%%%%%%%%%%%%%%%%%%%%%%%%%%%%%%%%%%%%%%%%%%%%%%%%%%

% 조건부 확률 문제
\begin{edXsection}{HW Exercise 2}

\begin{edXvertical}

\begin{edXproblem}{4-Door Monty Hall}

지난 수업시간에 몬티홀 문제를 다루었다. 이번에는 규칙을 다소 바꾸어서, 아래와 같은 방식으로 게임을 진행하기로 하자.
\begin{itemize}
\item 참가자가 고를 수 있는 문이 4개이다.
\item 참가자는 4개의 문 중에서 선물이 있을 문을 선택한다.
\item 수업시간때와 마찬가지로 사회자는 참가자가 고르지 않은 방중에서 선물이 있지 않은 문을 보여준다. 
\end{itemize}
다음 질문에 대답하시오.

\begin{enumerate}
\item 참가자 마이클은 원래 선택한 문을 바꾸지 않기로 했다. 마이클이 선물을 획득할 확률은?
\edXabox{expect="0.25" otype="numerical" tolerance='0.01'}

\item 다른 참가자 강호동은 선택을 바꾸어 나머지 두 개의 문중에서 같은 확률로 새로운 문을 고르기로 했다. 
강호동이 선물을 획득할 확률은?
\edXabox{expect="0.375" type="numerical" tolerance='0.001'}

\item 문제의 규칙을 다시 바꾸어서 이번에는 각각의 문에 1, 2, 3, 4 번호를 부여한 후, 
사회자는 선물이 있지 않은 문을 보여줄 때 가장 적은 숫자의 문을 보여주기로 한다. 
이 사실을 아는 참가자 강호동은 영리하게 선택을 바꾸고 이때 선택할 수 있는 문중에서 가장 적은 숫자의 문을 선택하기로 했다. 
강호동이 선물을 획득할 확률은?
\edXabox{expect="0.5" type="numerical" tolerance='0.1'}
\end{enumerate}
\end{edXproblem}

\begin{edXproblem}{Dice-Card}
주머니 A에는 1, 2, 3, 4, 5의 숫자가 하나씩 적혀 있는 5장의 카드가 들어 있고, 주머니 B에는 1, 2, 3, 4, 5, 6의 숫자가 
하나씩 적혀 있는 6장의 카드가 들어 있다. 한 개의 주사위를 한 번 던져서 나온 눈의 수가 3의 배수이면 주머니 A에서 임의로 
카드를 한장 꺼내고, 3의 배수가 아니면 주머니 B에서 임의로 카드를 한 장 꺼낸다. 
주머니에서 꺼낸 카드에 적힌 수가 짝수일 때, 그 카드가 주머니 A에서 꺼낸 카드일 확률은?
\end{edXproblem}

\begin{edXproblem}{Multiplication Theorem}
조건부 확률의 기본 공식 $(P(B|A) = \frac{P(A \cap B)}{P(A)}$ 과 귀납법(induction)을 사용하여,
다음 식을 증명하시오.
\begin{equation}
P(E_1 \cap E_2 \cap ... \cap E_n) = P(E_1) * P(E_2|E_1) * P(E_3 | E1 \cap E2)...*P(E_n | E_1 \cap E_2 \cap ... \cap E_{n-1})
\end{equation}
\end{edXproblem}

\end{edXvertical}
\end{edXsection}
\end{edXchapter}




 

 


