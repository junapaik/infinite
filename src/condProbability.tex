%%%%%%%%%%%%%%%%%%%%%%%%%%%%%%%%%%%%%%%%%%%%%%%%%%%%
\begin{edXchapter}{Conditional Probability}
%%%%%%%%%%%%%%%%%%%%%%%%%%%%%%%%%%%%%%%%%%%%%%%%%%%%

% 조건부 확률 문제
\begin{edXsection}{HW Exercise 2}

\begin{edXvertical}

\begin{edXproblem}{4-Door Monty Hall}

지난 수업시간에 몬티홀 문제를 다루었다. 이번에는 규칙을 다소 바꾸어서, 아래와 같은 방식으로 게임을 진행하기로 하자.
\begin{itemize}
\item 참가자가 고를 수 있는 문이 4개이다.
\item 참가자는 4개의 문 중에서 선물이 있을 문을 선택한다.
\item 수업시간때와 마찬가지로 사회자는 참가자가 고르지 않은 방중에서 선물이 있지 않은 문을 보여준다. 
\end{itemize}

\begin{enumerate}
\item 참가자 마이클은 원래 선택한 문을 바꾸지 않기로 했다. 마이클이 선물을 획득할 확률은?
\edXabox{expect="0.25" type="numerical" tolerance='0.01'}

\item 다른 참가자 강호동은 선택을 바꾸어 나머지 두 개의 문중에서 같은 확률로 새로운 문을 고르기로 했다. 강호동이 선물을 획득할 확률은?
\edXabox{expect="0.375" type="numerical" tolerance='0.001'}

\item 문제의 규칙을 다시 바꾸어서, 각각의 문에 1, 2, 3, 4 번호를 부여한 후, 사회자는 선물이 있지 않은 문을 보여줄 때 가장 적은 숫자의 문을 보여주기로 한다. 이 사실을 아는 참가자 강호동은 영리하게 선택을 바꾸고 이때 가장 적은 숫자를 가진 문을 선택하기로 했다. 강호동이 선물을 획득할 확률은?
\edXabox{expect="0.5" type="numerical" tolerance='0.1'}
\end{enumerate}
\end{edXproblem}

\end{edXvertical}
\end{edXsection}
\end{edXchapter}


 

 


