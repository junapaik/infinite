%%%%%%%%%%%%%%%%%%%%%%%%%%%%%%%%%%%%%%%%%%%%%%%%%%%%
\begin{edXchapter}{Definition of Probability}
%%%%%%%%%%%%%%%%%%%%%%%%%%%%%%%%%%%%%%%%%%%%%%%%%%%%

% 확률의 기초 Problem 1
\begin{edXsection}{HW Exercise}

\begin{edXvertical}

\begin{edXproblem}{Coin flippling}

동전의 앞면이 나올 확률과 뒷면이 나올 확률은 동일하다 가정하자. 동전을 한 번 던졌을때,
앞면이 나올 경우 개별사건을 {\tt H(=Head)}, 뒷면의 경우 {\tt T(=Tail)}로 표시하기로 하자.
동전을 10번 던졌을 때, 다음 질문에 대답하시오.


\begin{itemize}
\item 위 확률실험의 개별사건은 동전을 10번 던진 각각의 결과가 모인 것이라 볼 수 있다. 
(예를 들면, 앞면이 앞에 연속해서 5번, 뒷면이 뒤에 5번 나올 경우, {\tt HHHHHTTTTT} 로 표기) 
이럴 경우 표본공간$\Omega$ 의 크기(=$|\Omega|$) 를 구하시오?
\edXabox{expect="0.1" type="numerical" tolerance='0.001'}
\item 동전의 앞면과 뒷면이 동일한 숫자로 나올 확률은?
\edXabox{expect="0.2" type="numerical" tolerance='0.001'}
\item 앞면이 뒷면보다 많이 나올 확률은?
\edXabox{expect="0.3" type="numerical" tolerance='0.001'}
\item 4번 연속 앞면이 나올 확률은? (어느 위치에서 4번 연속 앞면이 나올지는 중요하지 않다.) 
\edXabox{expect="0.1" type="numerical" tolerance='0.001'}
\item 뒷면이 연속해서 마지막에 나올 확률은? (제일 마지막에 뒷면이 1번만 연속해서 등장하는 것도 이에 속하는 거로 간주한다).
\edXabox{expect="0.1" type="numerical" tolerance='0.001'}
\end{itemize}

\end{edXproblem}

% 조건부 확률 Problem 2
\begin{edXproblem}{Hit-And-Run Taxi}

한밤중에 택시 한 대가 뺑소니를 친 사건이 발생했다. 그 도시에는 두 개의 택시회사가 있는데,
청색사의 택시는 모두 청색이고 녹색사의 택시는 모두 녹색으로 칠해져 있다. 그 외에 주어진 
자료는 다음과 같다.
\begin{itemize}
\item 시내를 운행하고 있는 택시들 중 85\%는 녹색이고 나머지 15\%는 청색이다.
\item 한 목격자는 뺑소니차가 청색이었다고 증언했다. 법정은 신뢰성 있는 실험을 통해 동일한 상황에서,
정상 시력을 가진 사람이 차의 색상을 제대로 판별할 확률이 80\%라는 사실을 알아냈다. 즉 20\%의
사람들은 차의 색상을 잘못 판단한다는 뜻이다.
\end{itemize}

그렇다면 사고 당일 현장에서 뺑소니친 택시가 청색일 확률은?
\edXabox{expect="0.1" type="numerical" tolerance='0.001'}
\end{edXproblem}

\end{edXvertical}
\end{edXsection}
\end{edXchapter}









