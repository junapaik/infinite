%%%%%%%%%%%%%%%%%%%%%%%%%%%%%%%%%%%%%%%%%%%%%%%%%%%%
\begin{edXchapter}{Definition of Probability}
%%%%%%%%%%%%%%%%%%%%%%%%%%%%%%%%%%%%%%%%%%%%%%%%%%%%

% 확률의 기초 Problem 1
\begin{edXsection}{HW Exercise}

\begin{edXvertical}

\begin{edXproblem}{Coin flipping}

동전의 앞면이 나올 확률과 뒷면이 나올 확률은 동일하다 가정하자. 동전을 한 번 던졌을때,
앞면이 나올 경우 근원사건을 {\tt H(=Head)}, 뒷면의 경우 {\tt T(=Tail)}로 표시하기로 하자.
동전을 10번 던졌을 때, 다음 질문에 대답하시오.


\begin{itemize}
\item 위 시행의 근원사건은 동전을 10번 던진 각각의  결과가 모인 것이라 볼 수 있다(예를 들면, 앞면이 앞에 연속해서 5번, 뒷면이 뒤에 5번 나올 경우, {\tt HHHHHTTTTT} 로 표기한다).  
이 경우 표본공간$\Omega$ 의 크기(=$|\Omega|$) 는?
\edXabox{expect="1024" type="numerical" tolerance='1'}
\begin{edXsolution}
동전을 10번 던져서 나오는 모든 경우의 수는 $2^{10}$.
\end{edXsolution}
\item 동전의 앞면과 뒷면이 동일한 숫자로 나올 확률은?
\edXabox{expect="0.246" type="numerical" tolerance='0.01'}
\begin{edXsolution}
10개의 동전 중 5개를 고르는 경우의 수는 $10C_5$.
위의 문제에서 계산한 전체 표본공간의 크기는 1024이다. 그러므로 동전의 앞면이 5개 나올 사건의 확률은 $\frac{10C_5}{1024}$.
\end{edXsolution}
\item 앞면이 뒷면보다 많이 나올 확률은?
\edXabox{expect="0.377" type="numerical" tolerance='0.01'}
\begin{edXsolution}
앞면이 뒷면보다 많이 나올 경우는 앞면이 6개 7개 8개 9개 10개 나올 경우이므로, 
전체 경우의 수는 $10C_6 + 10C_7 + 10C_8 + 10C_9 + 10C_10 = 386 $ 이다.
따라서, 확률은 $\frac{386}{1024}$.
\end{edXsolution}
\item 4번 연속 앞면이 나올 확률은? (어느 위치에서 4번 연속 앞면이 나올지는 중요하지 않다.) 
\edXabox{expect="0.245" type="numerical" tolerance='0.001'}
\begin{edXsolution}
(이 문제는 어려운 문제이다. 문제 해설을 끝까지 읽어보자.) H와 T로 이루어진 n(>=5)개의 4개이상의 연속된 H를 포함하지 않는 방법을 생각해 보자.
먼저 k개의 문자열이 4개 이상의 연속된 H를 포함하지 않을 확률을 $P_k$라 정의해 두자.
이 문자열이 시작하는 방법은 다음 4가지가 있다.
\begin{enumerate}
\item T
\item HT
\item HHT
\item HHHT
\end{enumerate}
(주의: 이 네가지 방법들이 우리가 원하는 모든 경우를 나타내고 네 방법은 서로 겹치지 않음에 유의하자.)
(1)의 경우에 n개의 문자열이 4개 이상의 연속된 H를 포함하지 않으려면 T를 제외한 나머지 n-1 개의 문자열이
4개 이상의 연속된 H를 포함하지 않으면 된다. 첫번째에 T가 될 확률은 $\frac{1}{2}$이므로 (1)의 형태이고
n개의 문자열이 4개 이상 연속된 H를 포함하지 않을 확률은 $\frac{1}{2}P_{n-1}$이다. 

다른 경우들도 마찬가지로 생각하며 (2)의 경우 $\frac{1}{4}P_{n-2}$, (3)의 경우 $\frac{1}{8}P_{n-3}$,
(4)의 경우 $\frac{1}{16}P_{n-4}$이다. 따라서,
$P_n = \frac{1}{2}P_{n-1} + \frac{1}{4}P_{n-2} + \frac{1}{8}P_{n-3} + \frac{1}{16}P_{n-4}$ (n>=5)

$P_1 = P_2 = P_3 = 1$, $P4=\frac{15}{16}$이므로 $P5 = \frac{1}{2}\times\frac{15}{16} + \frac{1}{4}\times1 + \frac{1}{8}\times1 + \frac{1}{16}\times1 = \frac{29}{32}$
마찬가지로 $P_6$, $P_7$, ..., $P_10$을 구할 수 있다.
우리가 구하는 확률은 $1-P_{10} = \frac{251}{1024}$.
\end{edXsolution}
\item 뒷면이 2번이상 연속해서 마지막에 나올 확률은? 
\edXabox{expect="0.25" type="numerical" tolerance='0.001'}
\end{itemize}
\begin{edXsolution}
뒷면이 마지막 두번 연속해서 나올 확률은 $\frac{1}{2} \times \frac{1}{2} = \frac{1}{4}$ 이다.
\end{edXsolution}
\end{edXproblem}

% 조건부 확률 Problem 2
\begin{edXproblem}{Hit-And-Run Taxi}

한밤 중에 택시 한 대가 뺑소니를 친 사건이 발생했다. 그 도시에는 두 개의 택시회사가 있는데,
청색사의 택시는 모두 청색이고 녹색사의 택시는 모두 녹색으로 칠해져 있다. 그 외에 주어진 
자료는 다음과 같다.
\begin{itemize}
\item 시내를 운행하고 있는 택시들 중 85\%는 녹색이고 나머지 15\%는 청색이다.
\item 한 목격자는 뺑소니차가 청색이었다고 증언했다. 법정은 신뢰성 있는 실험을 통해 동일한 상황에서,
정상 시력을 가진 사람이 차의 색상을 제대로 판별할 확률이 80\%라는 사실을 알아냈다. 즉 20\%의
사람들은 차의 색상을 잘못 판단한다는 뜻이다.
\end{itemize}

그렇다면 사고 당일 현장에서 뺑소니친 택시가 청색일 확률은?
\edXabox{expect="0.414" type="numerical" tolerance='0.001'}
\begin{edXsolution}
택시가 청색인 사건을 B, 택시가 녹색인 사건을 G(= $B^c$), \newline
목격자가 색을 정확힌 본 사건을 T, 목격자가 색을 부정확하게 본 사건을 F(=$T^c$)
라 하자.

택시가 청색으로 보이는 사건은 $(B \cap T) \cup (G \cap F)$,
택시가 청색인 사건B는  $(B \cap T) \cup (B \cap F)$ 로 쓸 수 있고,
위 두사건의 교집합은 $(B \cup T)$이다.

구하는 확률은 
\begin{equation}
P(B |{(B \cup T) \cap (G \cup F)}) = \frac{P(B \cup T)}{P((B \cup T) \cap (G \cup F))} = \frac{0.15 \times 0.8}{0.15 \times 0.8 + 0.85 \times 0.2 = \frac{12}/{29}
\end{equation}
\end{edXsolution}
\end{edXproblem}

\end{edXvertical}
\end{edXsection}
\end{edXchapter}
