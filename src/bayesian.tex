%%%%%%%%%%%%%%%%%%%%%%%%%%%%%%%%%%%%%%%%%%%%%%%%%%%%
\begin{edXchapter}{Bayesian}
%%%%%%%%%%%%%%%%%%%%%%%%%%%%%%%%%%%%%%%%%%%%%%%%%%%%

% 확률의 기초 Problem 1
\begin{edXsection}{HW Exercise 3}

\begin{edXvertical}

\begin{edXproblem}{Weatherman}
베이즈 정리는 다음과 같다.
\begin{equation}
P(A|B) = \frac{P(B|A) \times P(A)}{P(B)} 
\end{equation}
\begin{enumerate}
\item 베이즈 정리를 증명하시오.

\item 기상캐스터인 A는  매일 걸어서 방송국으로 출근한다. 보통 비가 올 확률 {\tt\LARGE P(rain)} 은  0.30 이다.
A는 비가 내릴거라고 예상한 날 우산을 가져가기도 하지만, 햇볕이 강한 날은 자외선을 차단하기 
위해 우산을 가져가기도 한다. A가 우산을 가져 갈 확률은 {\tt\LARGE P(rain)}은 0.40 이다. 
기상캐스터답게 A씨는 보통 비를 잘 맞지 않는다. A가 비를 맞지 않을 확률은 {\tt\LARGE P(umbrella | rain)} 은 0.80 이다.
A가 우산을 들고 출근한 날, 비가 내렸을 확률은? (베이즈 정리를 사용하시오.)
\edXabox{expect="0.00" type="numerical" tolerance='0.01'}
\item 사건의 원인과 결과 관점에서 베이즈 정리의 의미를 서술하시오.
(\href{http://ko.wikipedia.org/wiki/베이즈_정리} 참조)
\end{enumerate}
\end{edXproblem}

\begin{edXproblem}{Bad Exam}
풍산고등학교 수학시험에는 늘 10 퍼센트의 오류가 있다. A 선생님은 문제에 오류가 있는 지 물었을 때,
80 퍼센트 정확히 대답한다 (문제에 오류가 없을때도 마찬가지이다). B 선생님은 75 퍼센트 정확히 대답한다.
다음과 같이 확률 사건을 정의하자.
\begin{itemize}
\item E := [문제에 오류가 있음]
\item T := [A 선생님이 문제에 오류가 있다고 진단함]
\item L := [B 선생님이 문제에 오류가 있다고 진단함]
\end{itemize}

다음 질문에 대답하시오.
\begin{enumerate} 
\item 풍산고등학교 학생 C는 시험 문제가 잘못된 것인지 의심되어, 
A 선생님에게 문제에 오류가 있는지 물었더니, A선생님은 '오류가 없다'라고 대답했다.
한번 더 확인하기 위해 B 선생님에게 물었더니, 오류가 있다라고 대답하였다.
실제 문제에 오류가 있을 확률은?
\edXabox{expect="0.00" type="numerical" tolerance='0.01'}
\item 사건 T는 사건 L과 독립사건인가? (곧, {\tt\LARGE P(T|L) = P(T)})
\edXabox{expect="0.00" type="numerical" tolerance='0.01'}
\end{enumerate}
\end{edXproblem}

\begin{edXproblem}{Same Modulo3 Stickers}
주머니 안에 스티커가 1개, 2개, 3개 붙어 있는 카드가 각가 1장씩 들어 있다. 주머니에서 임의로
카드 1장을 꺼내어 스티커 1개를 더 붙인 후 다시 주머니에 넣는 시행을 반복한다. 주머니 안의
각 카드에 붙어 있는 스티커의 개수를 3으로 나눈 나머지가 모두 같으지는 사건을 A라고 하자.
시행을 6번 하였을 때, 1회부터 5회까지는 사건 A가 일어나지 않고, 6회에서 사건 A가 일어날 
확률을 {\tt\LARGE\frac{q}{p}}라고 하자. {\tt\LARGEp+q}의 값을 구하시오 (p와 q는 서로소인 자연수이다)
\edXabox{expect="0.00" type="numerical" tolerance='0.01'}
\end{edXproblem}

\end{edXvertical}
\end{edXsection}
\end{edXchapter}
